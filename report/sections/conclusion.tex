\section{Conclusion}\label{sec:conclusion}
In this project, we successfully implemented the Proximal Policy Optimization (PPO) algorithm and evaluated its performance on a diverse set of Reinforcement Learning environments: CartPole-v1, HalfCheetah-v5, and Reacher-v5. Our implementation incorporated key PPO features, including an actor-critic architecture with optional shared layers, Generalized Advantage Estimation (GAE), a clipped surrogate objective, multiple update epochs, and an entropy bonus for exploration.

The experimental results demonstrate that our PPO agent effectively learns policies for both discrete and continuous control tasks. In the CartPole-v1 environment, the agent quickly achieved optimal performance, mastering the task to a perfect reward of \textbf{500.0} within 20,000 timesteps. For the more complex MuJoCo environments, HalfCheetah-v5 and Reacher-v5, our agent achieved average rewards of approximately \textbf{1770.62} and \textbf{-8.45}, respectively, after 1 million timesteps. These results are consistent with the performance benchmarks reported in the original PPO paper~\cite{schulman2017proximal}, validating the correctness and effectiveness of our implementation.

Furthermore, we developed an interactive Streamlit application to visualize the behavior of the trained agents. This tool allows for on-the-fly rendering of agent performance, providing a qualitative understanding of the learned policies and enhancing the interpretability of the results.

Future work could explore several directions. Firstly, extending the range of environments to include more complex tasks or different domains (e.g., pixel-based observations) would further test the robustness and scalability of the implementation. Secondly, a more extensive hyperparameter optimization study could potentially lead to improved performance on the current set of tasks. Investigating the impact of different network architectures, such as using different activation functions or layer configurations, could also yield performance benefits. Finally, incorporating more advanced techniques, such as curiosity-driven exploration or recurrent policies for partially observable environments, could expand the capabilities of the agent.

In summary, this project provides a solid implementation of the PPO algorithm, demonstrates its successful application to standard benchmark tasks, and offers a foundation for further research and development in the field of reinforcement learning.